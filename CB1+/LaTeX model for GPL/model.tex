%A LaTeX model for SYSU General Phys. Lab. by Probfia Gao.
%用XeLaTeX编译
\documentclass[11pt,a4paper]{ctexart}

%在下面补全实验名,例如 实验BB3 光电效应实验。
\newcommand{\ExpeName}{实验CB1+ 迈克尔逊干涉及应用(白光干涉)}

\usepackage{fancyhdr}
\usepackage{amsmath}
\usepackage{amssymb}
\usepackage{graphicx}
\usepackage[hmargin=1.25in,vmargin=1in]{geometry}
\usepackage{pdfpages}
\usepackage[colorlinks,
            linkcolor=red,
		 urlcolor=black]{hyperref}
\usepackage{cleveref}
\usepackage{float}

\crefname{equation}{}{}
\crefname{figure}{图}{图}
\crefname{footnote}{注释}{注释}
\crefname{table}{表}{表}

%\cpic{<尺寸>}{<文件名>}}用于生成居中的图片。
\newcommand{\cpic}[2]{
\begin{center}
\includegraphics[scale=#1]{#2}
\end{center}
}

%\cpicn{<尺寸>}{<文件名>}{<注释>}用于生成居中且带有注释的图片,其label为图片名。
\newcommand{\cpicn}[3]
{
\begin{figure}[H]
\cpic{#1}{#2}
\caption{#3\label{#2}}
\end{figure}
}

\newcommand{\beq}{\begin{equation}}
\newcommand{\eeq}{\end{equation}}
\newcommand{\bea}{\begin{equation}\begin{aligned}}
\newcommand{\eea}{\end{aligned}\end{equation}}

%输入单位和数学常数
%下面所有命令需在公式环境下使用
\newcommand{\e}{\mathrm{\ e}}   %自然常数e = \e
\newcommand{\im}{\mathrm{\ i}}   %虚数单位i = \im
\newcommand{\meter}{\mathrm{\ m}}      %单位/前缀 = \单位/前缀英文名
\newcommand{\newton}{\mathrm{\ N}}  
\newcommand{\joule}{\mathrm{\ J}}
\newcommand{\second}{\mathrm{\ s}}
\newcommand{\gram}{\mathrm{\ g}}
\newcommand{\ampere}{\mathrm{\ A}}
\newcommand{\kilogram}{\mathrm{\ kg}}
\newcommand{\kelvin}{\mathrm{\ K}}
\newcommand{\mole}{\mathrm{\ mol}}
\newcommand{\volt}{\mathrm{\ V}}
\newcommand{\degreeC}{\ ^\circ \mathrm{C}}  %摄氏度符号 = \degreeC


\newcommand{\emptyline}{\par \ \\}


\pagestyle{fancy}

\fancyhead[L]{\footnotesize{中山大学物理与天文学院基础物理实验}}
\fancyhead[R]{\footnotesize{\ExpeName}}
\fancyfoot[C]{\thepage}

\begin{document}
%第一页
\cpic{0.255}{e1}%学生信息和计分表格
\begin{center}
\LARGE\textbf{{\ExpeName}}
\end{center}
\large{【实验报告注意事项】}
\begin{enumerate}
 \item 实验报告由三部分组成:
 \begin{enumerate}
  \item[1)]预习报告:(提前一周)认真研读\textbf{\uline{实验讲义}},弄清实验原理;实验所需的仪器设备、用具及其使用(强烈建议到实验室预习),完成讲义中的预习思考题;了解实验需要测量的物理量,并根据要求提前准备实验记录表格(由学生自己在实验前设计好,可以打印)。预习成绩低于10分(共20分)者不能做实验。
  \item[2)]实验记录:认真、客观记录实验条件、实验过程中的现象以及数据。实验记录请用珠笔或者钢笔书写并签名({\color{red}用铅笔记录的被认为无效})。{\color{red}保持原始记录,包括写错删除部分,如因误记需要修改记录,必须按规范修改。}(不得输入电脑打印,但可扫描手记后打印扫描件);离开前请实验教师检查记录并签名。
  \item[3)]分析讨论:处理实验原始数据(学习仪器使用类型的实验除外),对数据的可靠性和合理性进行分析;按规范呈现数据和结果(图、表),包括数据、图表按顺序编号及其引用;分析物理现象(含回答实验思考题,写出问题思考过程,必要时按规范引用数据);最后得出结论。
 \end{enumerate}
 \textbf{实验报告}就是预习报告、实验记录、和数据处理与分析合起来,加上本页封面。
 \item 每次完成实验后的一周内交\textbf{实验报告}。
 \item 除实验记录外,实验报告其他部分建议双面打印。
\end{enumerate}
\ 
\\
\ 

\begin{flushright}                                                           %模板作者
\tiny{
A \LaTeX \ model for General Phys. Lab., SPA, SYSU by {\em \href{https://www.weibo.com/3532532974/profile?rightmod=1&wvr=6&mod=personinfo&is_all=1}{Probfia} Gao.}\\ Adopted from the \href{http://lovephysics.sysu.edu.cn/lib/exe/fetch.php?media=courses:secondlevelzhuhai:report.docx}{original MS Word model} on \href{http://lovephysics.sysu.edu.cn}{Lovephysics}.\\ You can view it on \href{https://github.com/Probfia/SYSU_GPL_C}{Github}.}
\end{flushright}

\newpage%预习报告
\begin{center}
\LARGE{\textbf{\ExpeName}}
\end{center}
\textbf{【实验目的】}
\begin{enumerate}
 \item[1.] 观察等倾、等厚干涉现象及调节白光干涉条纹;
 \item[2.] 学习用迈克尔逊干涉仪测量钠光谱波长差的方法;
 \item[3.] 学习用白光干涉测量透明薄片折射率的方法;
 \item[4.] 用迈克尔逊干涉仪测量多种光源的相干长度
\end{enumerate}
\textbf{【仪器用具】}
%将讲义中的表格截图保存为t1在该文件夹下后删去下一行之前的%符号,合理调整scale参数。
\cpic{0.3}{t1}
%或者自己去 https://www.tablesgenerator.com/ 做一个表。
\textbf{【原理概述】}\par
该实验观察复色光的等倾干涉现象。在迈克尔逊干涉仪中,等倾干涉的光程差为
\beq
L = \sqrt{(Z+2d)^2 + R^2} - \sqrt{Z^2 + R^2} = \sqrt{Z^2 + R^2}(\sqrt{1+\frac{4d(Z+d)}{Z^2 + R^2}} - 1) \simeq 2d\cos \theta
\eeq
当光程差为波长的整数倍时,出现明纹;半奇数倍时,出现暗纹。明纹条件是
\beq
\theta = \arccos \frac{m \lambda}{2d}
\eeq
但对于复色光来说,波长非单一,因此,干涉仪会将不同波长的光分开,形成彩色条纹。但彩色条纹中间依然存在一个暗纹,称为中心暗纹。观察到中心暗纹后,移动$M_1$镜使得中心暗纹移到视场中央,在$M_1$镜和分束镜$P_1$间放上折射率为$n$,厚度为$t$的透明薄片,且让薄片和$M_1$镜平行,光程差就相应地扩大了$\Delta L = 2t(n-1)$。这个光程差使得彩色条纹移出视场范围,此时再调整$M_1$镜的位置$\Delta d = \frac{1}{2} \Delta L$,就可以使得彩色条纹重新出现。折射率于是为
\beq
n = \frac{\Delta d}{t} + 1
\eeq
\par
该实验的另外一部分是测量钠双线的波长差。钠黄光含有两种波长相近的光($\lambda_{1}=589.0 \mathrm{\ nm}, \quad \lambda_{2}=589.6 \mathrm{\ nm}$)。采用钠灯作光源时,两
条谱线形成各自的干涉条纹,在视场中的两套干涉条纹相互叠加。由于波长不同,同级条纹
之间会产生错位($\lambda_1$)的某一级的暗条纹可能会和 λ2的另一级的亮条纹重合)。在移动反射镜$M_1$(光程差发生变化)过程中,干涉条纹会出现清晰与模糊的周期性变化,称为“光拍现
象”。其原理见讲义。\par
当条纹发生“清晰-模糊-清晰”变化现象时,反射镜移动$\Delta d$的距离,波长差就是
\beq
\Delta \lambda = \frac{\bar{\lambda}^2}{2\Delta d}
\eeq
\par
该实验的最后一部分是测量汞灯的相干长度。我们知道,一束光的相干长度是它能够保持相干性的最大长度,于是有
\beq
l_c \sim \frac{c}{\Delta f} = \frac{\lambda^2}{\Delta \lambda}
\eeq
为了测量相干长度,我们从0开始改变光程差,直到干涉条纹变得十分模糊,便是相干长度的测量值
\beq
l_c \sim 2\Delta d
\eeq

\textbf{【实验前思考题】}
\begin{enumerate}
 \item[1.] (问题1)
 \item[2.] (问题2)
\end{enumerate}

\newpage%实验记录
\cpic{0.255}{e2}%学生信息表格
\begin{center}
\LARGE{\textbf{\ExpeName}}
\end{center}
\textbf{【实验内容、步骤、结果】}
1.\textbf{测量钠双黄线的波长差}\par
记录出现“模糊-清晰-模糊”现象的测微头读数如\cref{table2}
\begin{table}[H]
    \centering
    \caption{钠双黄线波长差的测量}
    \label{table2}
    \begin{tabular}{|c|p{20mm}|p{20mm}|p{20mm}|}
    \hline
    第一次模糊时的测微头位置$d_1/\mathrm{mm}$ &   &    & \\ \hline
    第二次模糊时的测微头位置$d_2/\mathrm{mm}$ &   &    & \\ \hline
    \end{tabular}
\end{table}
\emptyline
2.\textbf{透明薄片的折射率}
\par
先测量薄片的厚度,数据记录入\cref{table3}
\begin{table}[H]
    \centering
    \caption{薄片厚度的测量}
    \label{table3}
    \begin{tabular}{|p{20mm}|p{20mm}|p{20mm}|}
    \hline
    $t_1$ & $t_2$ & $t_3$ \\ \hline
     &  &  \\ \hline
    \end{tabular}
\end{table}
\par
调节测微头出现全黑条纹,记下此时的读数$d_1$,再加上薄片,调节测微头使得全黑条纹重现,记下此时的读数$d_2$。数据记入\cref{table1},重复3次。
\begin{table}[H]
    \centering
    \caption{测微头移动距离的测量}
    \label{table1}
    \begin{tabular}{|c|p{20mm}|p{20mm}|p{20mm}|}
    \hline
    $d_1/\mathrm{mm}$ &   &    & \\ \hline
    $d_2/\mathrm{mm}$ &   &    & \\ \hline
    \end{tabular}
\end{table}
\emptyline
3.\textbf{测量汞灯的相干长度}\par
从0开始改变光程差,直到干涉条纹完全模糊,记录下此时测微头的读数,如\cref{table4}
\begin{table}[H]
    \centering
    \caption{汞灯干涉长度的测量}
    \label{table4}
    \begin{tabular}{|p{20mm}|p{20mm}|p{20mm}|}
    \hline
    $d_1$ & $d_2$ & $d_3$ \\ \hline
     &  &  \\ \hline
    \end{tabular}
\end{table}

\emptyline
\textbf{【实验过程中遇到问题记录】}

%生成最终报告时将上面内容全部删除或注释(用\iffalse \fi),将扫描得到的实验报告保存为Record.pdf在LaTeX model for GPL下,将下行命令的注释号删去。注意根据实际页数调整pages参数。
%\includepdf[pages=1-3]{Record}

\newpage%分析与讨论
\cpic{0.255}{e3}%学生信息表格
\begin{center}
\LARGE\textbf{{\ExpeName}}
\end{center}
\textbf{【分析与讨论】}\par
(Content)
\newline
\textbf{【实验思考题】}\par
(Content)

\end{document}