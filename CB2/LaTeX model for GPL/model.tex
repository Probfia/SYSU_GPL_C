%A LaTeX model for SYSU General Phys. Lab. by Probfia Gao.
%用XeLaTeX编译
\documentclass[11pt,a4paper]{ctexart}

%在下面补全实验名,例如 实验BB3 光电效应实验。
\newcommand{\ExpeName}{实验CB2 塞曼效应}

\usepackage{fancyhdr}
\usepackage{amsmath}
\usepackage{graphicx}
\usepackage[hmargin=1.25in,vmargin=1in]{geometry}
\usepackage{pdfpages}
\usepackage[colorlinks,
            linkcolor=red,
		 urlcolor=black]{hyperref}
\usepackage{cleveref}

\crefname{equation}{}{}
\crefname{figure}{图}{图}
\crefname{footnote}{注释}{注释}
\crefname{table}{表}{表}

%\cpic{<尺寸>}{<文件名>}}用于生成居中的图片。
\newcommand{\cpic}[2]{
\begin{center}
\includegraphics[scale=#1]{#2}
\end{center}
}

%\cpicn{<尺寸>}{<文件名>}{<注释>}用于生成居中且带有注释的图片,其label为图片名。
\newcommand{\cpicn}[3]
{
\begin{figure}[h!]
\cpic{#1}{#2}
\caption{#3\label{#2}}
\end{figure}
}

\newcommand{\beq}{\begin{equation}}
\newcommand{\eeq}{\end{equation}}
\newcommand{\bea}{\begin{equation}\begin{aligned}}
\newcommand{\eea}{\end{aligned}\end{equation}}

%输入单位和数学常数
%下面所有命令需在公式环境下使用
\newcommand{\e}{\mathrm{\ e}}   %自然常数e = \e
\newcommand{\im}{\mathrm{\ i}}   %虚数单位i = \im
\newcommand{\meter}{\mathrm{\ m}}      %单位/前缀 = \单位/前缀英文名
\newcommand{\newton}{\mathrm{\ N}}  
\newcommand{\joule}{\mathrm{\ J}}
\newcommand{\second}{\mathrm{\ s}}
\newcommand{\gram}{\mathrm{\ g}}
\newcommand{\ampere}{\mathrm{\ A}}
\newcommand{\kilogram}{\mathrm{\ kg}}
\newcommand{\kelvin}{\mathrm{\ K}}
\newcommand{\mole}{\mathrm{\ mol}}
\newcommand{\volt}{\mathrm{\ V}}
\newcommand{\degreeC}{\ ^\circ \mathrm{C}}  %摄氏度符号 = \degreeC


\newcommand{\emptyline}{\\ \ \\}

\pagestyle{fancy}

\fancyhead[L]{\footnotesize{中山大学物理与天文学院基础物理实验}}
\fancyhead[R]{\footnotesize{\ExpeName}}
\fancyfoot[C]{\thepage}

\begin{document}
%第一页
\cpic{0.255}{e1}%学生信息和计分表格
\begin{center}
\LARGE\textbf{{\ExpeName}}
\end{center}
\large{【实验报告注意事项】}
\begin{enumerate}
 \item 实验报告由三部分组成:
 \begin{enumerate}
  \item[1)]预习报告:(提前一周)认真研读\textbf{\uline{实验讲义}},弄清实验原理;实验所需的仪器设备、用具及其使用(强烈建议到实验室预习),完成讲义中的预习思考题;了解实验需要测量的物理量,并根据要求提前准备实验记录表格(由学生自己在实验前设计好,可以打印)。预习成绩低于10分(共20分)者不能做实验。
  \item[2)]实验记录:认真、客观记录实验条件、实验过程中的现象以及数据。实验记录请用珠笔或者钢笔书写并签名({\color{red}用铅笔记录的被认为无效})。{\color{red}保持原始记录,包括写错删除部分,如因误记需要修改记录,必须按规范修改。}(不得输入电脑打印,但可扫描手记后打印扫描件);离开前请实验教师检查记录并签名。
  \item[3)]分析讨论:处理实验原始数据(学习仪器使用类型的实验除外),对数据的可靠性和合理性进行分析;按规范呈现数据和结果(图、表),包括数据、图表按顺序编号及其引用;分析物理现象(含回答实验思考题,写出问题思考过程,必要时按规范引用数据);最后得出结论。
 \end{enumerate}
 \textbf{实验报告}就是预习报告、实验记录、和数据处理与分析合起来,加上本页封面。
 \item 每次完成实验后的一周内交\textbf{实验报告}。
 \item 除实验记录外,实验报告其他部分建议双面打印。
\end{enumerate}
\ 
\\
\ 

\begin{flushright}                                                           %模板作者
\tiny{
A \LaTeX \ model for General Phys. Lab., SPA, SYSU by {\em \href{https://www.weibo.com/3532532974/profile?rightmod=1&wvr=6&mod=personinfo&is_all=1}{Probfia} Gao.}\\ Adopted from the \href{http://lovephysics.sysu.edu.cn/lib/exe/fetch.php?media=courses:secondlevelzhuhai:report.docx}{original MS Word model} on \href{http://lovephysics.sysu.edu.cn}{Lovephysics}.\\ You can view it on \href{https://github.com/Probfia/SYSU_GPL_C}{Github}.}
\end{flushright}

\newpage%预习报告
\begin{center}
\LARGE{\textbf{\ExpeName}}
\end{center}
\textbf{【实验目的】}
\begin{enumerate}
 \item[1.]通过观察原子谱线在外磁场中的分裂现象,加深对电子“自旋”、“两个角动量的耦合”、“两个电子之间的$ LS $耦合”、“角动量守恒”、“多电子原子和电子组态”、“能级跃迁”、“选择定则”等概念的理解,验证原子具有磁矩及其空间取向的量子化,进一步认识原子的内部结构;
 \item[2.]用F-P干涉仪观察汞(Hg)原子546.1 nm 谱线在外磁场中的分裂现象(即“反常塞曼效应”),测量电子的荷质比$ e/m$;
 \item[3.]验证光子具有角动量及角动量守恒定律,了解光的偏振理论及其产生机制,学习偏振片的原理及使用方法。
\end{enumerate}
\textbf{【仪器用具】}
%将讲义中的表格截图保存为t1在该文件夹下后删去下一行之前的%符号,合理调整scale参数。
\cpic{0.3}{t1}
%或者自己去 https://www.tablesgenerator.com/ 做一个表。
\textbf{【原理概述】}\par
该实验用F-P干涉仪观察汞的反常塞曼效应。\par
对原子施加沿$z$轴方向的外磁场$B$后,由于电子的轨道角动量和自旋角动量的$z$轴分量都是量子化的,施加的外磁场将导致哈密顿算符发生改变,原先简并的本征态发生分裂。汞的外层有2个电子,它们的轨道角动量量子数和自旋角动量量子数分别标记为$(l_1,m_1)$和$(l_2,m_2)$。两个电子的轨道角动量和自旋角动量分别相加,得到总轨道角动量$L$和总自旋角动量$S$,两者再相加得到总的角动量$J$,这种相加模式称为$LS$耦合。
\par
在这个实验中,汞原子有两种可能的电子组态,分别对应的各角动量量子数如\cref{table1}
\begin{table}[h!]
\centering
\caption{汞原子两种电子组态对应的量子数}
\label{table1}
\begin{tabular}{|c|c|c|}
\hline
 & $^3 S_1\ (6s7s)$ & $^3 P_2\ (6s7p)$ \\ \hline
$L$ & 0 & 1 \\ \hline
$S$ & 1 & 1 \\ \hline
$J$ & 1 & 2 \\ \hline
\end{tabular}
\end{table}
\par
磁矩$\vec{J}$在外磁场$B$中具有额外的势能
\beq
U = -\vec{\mu}_J \cdot \vec{B}
\eeq
或者用角动量$J$的量子数表示为
\beq \label{addeng}
U = m_J g_J \mu_B B
\eeq
若电子在主量子数$n$的不同值间跃迁的发射频率为$\nu$,加入外磁场后,原子具有了由\cref{addeng}给出的额外能量,从而带来了额外的频率。这一频率值是
\beq
\Delta \nu = (M_2 g_2 - M_1 g_1) \frac{eB}{4 \pi m} = 46.7 (M_2 g_2 - M_1 g_1) c B \ (\mathrm{Hz})
\eeq
\par
塞曼效应中发射的光子有两种偏振模式:垂直于磁场的$\sigma^\pm$偏振和平行于磁场的$\pi$偏振。$\sigma^\pm$偏振是垂直磁场平面的圆偏振光,其角动量方向与磁场平行或相反;$\pi$偏振前后原子角动量不变,光子角动量平行于磁场,是与磁场平行的线偏振光。
\emptyline
\textbf{【实验前思考题】}
\begin{enumerate}
 \item[1.] \textbf{光子是否具有角动量?试描述光子角动量方向与光的偏振方向之间的关系。}\par
光子具有角动量,其方向与其圆偏振方向满足右手螺旋定则关系。
 \item[2.] \textbf{用同一级条纹的内外圈分别计算电子的荷质比,结果一样吗?试简述原因。}\par
 \item[3.] \textbf{请利用(20)至(23)式,计算汞原子 3S1(6s7s)和 3P2(6s7p)能级所对应的量子数(见表 1),并给出详细的计算过程。}\par
对$^3S$组态,$L = 0$,$S=1$,$J$的可能取值从$|L-S|$到$L + S$,其值都是1,故$J$只能取1  。\par
对$^3P$组态,$L = 1$,$S = 1$,$j$的可能取值为$J = 0,1,2$,表中给出的是$J = 2$的情况。
 \item[4.] \textbf{请利用(2)、(8)和(20)式,并结合$\vec{J} = \vec{L} + \vec{S}$和$\vec{\mu}_J = \vec{\mu}_L + \vec{\mu}_S$(注意此时的$\vec{\mu}_J$是图 5 中的$\vec{\mu}_J$,详细见脚注 22),导出朗德因子的一般表达式(28)式,并给出详细的推导过程。}\par
由$\vec{\mu}_L = -\gamma \vec{L}$和$\vec{\mu}_S = -2\gamma \vec{S}$和
\beq
\vec{\mu}_J = \vec{\mu}_L +\vec{\mu}_S
\eeq
并注意到$\vec{J} = \vec{L} + \vec{S}$,有
\beq
-\vec{\mu}_J \equiv g_J \gamma \vec{J} = \gamma \vec{L} + 2\gamma \vec{S}
\eeq
也即
\beq
g_J \vec{J} = \vec{L} + 2\vec{S}
\eeq
上式两边点乘$\vec{J}$得到
\bea
g_J \vec{J}^2 &= \vec{L} \cdot \vec{J} + 2\vec{S} \cdot \vec{J} \\
&= \vec{L}^2 + \vec{S}\cdot \vec{L}  + 2\vec{S}^2 + 2\vec{S} \cdot \vec{L} \\
&= \vec{L}^2 + \frac{3}{2} [(\vec{L} + \vec{S})^2 - \vec{L}^2 - \vec{S}^2] + 2\vec{S}^2 \\
&= \frac{3}{2} \vec{J}^2 + \frac{1}{2}(\vec{S}^2 - \vec{L}^2) 
\eea
再注意到$\vec{J}^2 = J(J+1)\hbar^2$,$\vec{L}^2 = L(L+1)\hbar^2$,$\vec{S}^2 = S(S+1)\hbar^2$,有
\beq
g_J J(J+1) = \frac{3}{2} J(J+1) + \frac{1}{2}[S(S+1)- L(L+1)]
\eeq
也即
\beq
g_J = \frac{3}{2} - \frac{L(L+1) - S(S+1)}{2J(J+1)}
\eeq
\item[5.] \textbf{请利用单电子情况下的(36)式,并结合钠双黄线的平均波长及其波长差($\lambda_1 = 589.0 \mathrm{\ nm}$,$\lambda_2 = 589.6 \mathrm{\ nm}$),估算一下钠原子内部的磁感应强度$B_{int}$的值(提示:单电子情况下,两谱线的能级差为势能的两倍,即有;另需要利用到光子波长和频率之间的关系式。答案:钠原子内部的磁感应强度$B_{int}$的值为18.5 T)。}\par
我们有
\bea
B_{int} &= \frac{1.5\Delta \nu}{46.7} \\
&= \frac{1.5}{\Delta (\cfrac{1}{\lambda}) \times 46.7} \\
& = 18.5\mathrm{\ (\mu T)}
\eea

\item[6.] \textbf{请结合第 3 题的计算结果,说明弱外磁场$B_{ext}<<B_{int}$成立时弱外磁场$B_{ext}$的取值范围,并确认本实验中电磁体的磁感应强度符合弱外磁场$B_{ext} \ll B_{int}$条件。}
\item[7.] \textbf{请结合力与势能的关系式$\vec{F} \equiv - \nabla U$并利用(11)式,试推导磁矩在非均匀外磁场中的受力大小为$F_z = \mu_z \cfrac{\partial B_z}{\partial z} \ (B_x = B_y = 0)$(设外磁场方向在$z$轴方向,$F_z$为力在$z$方向上分量的大小)(提示:请利用郭硕鸿《电动力学》(第二版)一书附录中的矢量运算公式)。}\par
磁矩$\vec{\mu}$在磁场$\vec{B}$中具有势能
\beq
U = -\vec{\mu} \cdot \vec{B}
\eeq
在这里,$\vec{B} = B_z \vec{e}_z$,有$U = - \mu_z B_z$,于是受力为
\bea
\vec{F} &= - \frac{\partial U}{\partial z} \vec{e}_z \\
&=\mu_z \frac{\partial B_z}{\partial z} \vec{e}_z
\eea
也即
\beq
F_z = \mu_z \frac{\partial B_z}{\partial z}
\eeq

\item[8.] \textbf{请结合朗德因子的一般表达式(28)式,以及两个角动量耦合的一般规则(20)至(23)式,计算表 3 中汞原子 546.1nm 谱线对应的上下两个能级的各量子数及不同谱线(能级跃迁)的朗德因子(见图 9)。用“格罗春图”33(Grotrain 图)来表示汞原子 546.1nm谱线不同能级之间可能的跃迁。}

\item[9.] \textbf{请回答什么是“反常塞曼效应”和“正常塞曼效应”,两者之间的区别是什么。请思考什么是“帕邢-巴克效应”及其形成的原因。}\par
反常塞曼效应在电子净自旋为半奇数时发生,这时能级分裂为偶数个;正常塞曼效应则相反,在电子净自旋为整数时发生,这时能级分裂为奇数个。当外部磁场强于原子内部磁场时,电子间的耦合被破坏,使得谱线重新排列,这就是帕邢-巴克效应。

\item[10.] \textbf{请回答电子的“自旋—轨道耦合”的本质是什么?它与电子之间的“LS 耦合”的区别是什么?}
\end{enumerate}

\newpage%实验记录
\cpic{0.255}{e2}%学生信息表格
\begin{center}
\LARGE{\textbf{\ExpeName}}
\end{center}
\textbf{【实验内容、步骤、结果】}
\par
本实验自动测量和记录数据,结果由界面截图给出。
\newline
\textbf{【实验过程中遇到问题记录】}

%生成最终报告时将上面内容全部删除或注释(用\iffalse \fi),将扫描得到的实验报告保存为Record.pdf在LaTeX model for GPL下,将下行命令的注释号删去。注意根据实际页数调整pages参数。
%\includepdf[pages=1-3]{Record}

\newpage%分析与讨论
\cpic{0.255}{e3}%学生信息表格
\begin{center}
\LARGE\textbf{{\ExpeName}}
\end{center}
\textbf{【分析与讨论】}\par
(Content)
\newline
\textbf{【实验思考题】}\par
(Content)

\end{document}