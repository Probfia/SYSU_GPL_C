\documentclass[a4paper,11pt]{ctexart}
\usepackage[hmargin=1.25in,vmargin=1in]{geometry}
\author{高寒,物理与天文学院,17353019}
\date{}
\title{《共产党宣言》读书报告}
\begin{document}
\maketitle
《共产党宣言》的写就至今已经过去160年有余了。在我看来,宣言写就的目的,大概来讲,就是号召当时全世界受压迫的、被枷锁束缚的工人阶级联合起来,为争取自己的权益而不惜手段地与资产阶级斗争。社会是在不断运动的。在今天,马克思恩格斯的愿望是否实现?《共产党宣言》中的主要概念和思想在今天看来又应该如何被理解和看待?今天广大劳动者的现状又有何差异?在这篇读书报告中,我希望就宣言原文内容,结合当代国内外社会现状,特别是宣言中谈到的所有制问题,谈谈我自己粗浅的理解和看法。\par
《共产党宣言》是马克思和恩格斯于1847年年末与1848年年初写就并发表的,它是世界共产主义同盟的纲领性的、指导性的文件。在宣言的开头,马克思便直截了当地指出,一切历史都是阶级斗争的历史。在过去,由于封建的关系不再适应生产力的需要,对生产产生阻碍作用而非促进作用时,就自然而然地产生了资产阶级革命,例如英国的圈地运动,最终封建的约束被炸毁了。而在那个时代,世界面临的情况是,资本越加集中与少数的资本家手中,工人阶级面临着残酷的剥削和压迫,夜以继日的工作也难以维生。事实上,资产阶级的所有制以及资产阶级的各种关系显得太过狭隘,发达的生产力与资产阶级集中生产资料的事实相冲突,成为了对资产力进一步发展的阻碍,使得资产阶级制度的矛头重新指向资产阶级自己。\par
因此资产阶级理应以如封建所有制被推翻的原因那样被推翻。\par
十九世纪30至40年代,德国、英国和法国相继爆发了多次工人运动,旨在争取自己的政治权利,作为独立力量登上政治舞台。现在面对的事实是,无产阶级最终联合起来形成集体,但由于资本家的险恶用心,联合都是松散的,工人们常因为自己的短暂利益而变得争锋相对。同时,英法两国的空想社会主义或共产主义又都以失败告终,表明无产阶级运动需要有先进的理论指导。在1848年《共产党宣言》发表之时,正好遇上欧洲革命爆发,于是特别地,共产党强调整个无产阶级的共同利益,而非局限于单个个体、群体或单个民族的利益。共产党的目的概括起来就是,“消灭私有制”。\par
在《共产党宣言》中,马克思指出,“资本是集体的产物”。资本不是由单个人创造的,而是由社会中所有成员的共同劳动得来的。在这里的共同劳动中,工人阶级起到了重要的作用。例如,单有机器而无操作者,我们什么也创造不了。于是,资本家就雇佣工人为他们劳动,操作那些机器。作为劳动的报酬,资本家给以工人工资报酬,但在当时,这样的报酬,对于大多数劳动者而言,连勉强维生的难以达到。资本家希望通过雇佣工人来创造更多的财富,扩大产业,但因为劳动者苦不堪言的生活现状,这种雇佣关系反而限制了生产力的进一步发展。于是看来,为了解放生产力,就必须消灭这种披着雇佣外衣的,资产阶级对工人阶级的剥削。\par
在雇佣关系下,无产者的劳动不能为自己创造财产,而是为资产阶级创造了资本,这是剥削带来的财富。也就是说,本应当被创造者共同享有的资本,却最终因为资产阶级劳动关系,落到了少数资本家手里。因此宣言中提到,一般理解中的私有制,或私有财产,对约百分之九十的社会成员来说都早已不复存在,它们都早就被资产阶级日益壮大的过程给消灭了。对后继的无产阶级而言,要解放生产力,就要对资产阶级做它们曾经对封建阶级和小资产阶级做过的同样性质的事:消灭资产阶级私有制,“消灭那种以社会上的绝大多数人没有财产为必要条件的所有制”。\par
让我们把视角换到现在来想一想,这个目的,现在达到了吗?恐怕还没有。对西方国家,资产阶级代议制依然实行,不过工人阶级也可以组成政党加入议会,参与政治舞台,但就我们的观点来看,雇佣关系依然存在,资产阶级也没有丝毫要被消灭的兆头。而在中国,我们虽然是工人阶级政党领导下的国家,但在市场经济下,我们依然有着各式各样的雇佣关系,国家的基尼系数甚至高过许多西方国家,体现出财富和资本分布的极度不均衡。当然,这个问题在我们还在吃大锅饭的时代或许并不存在,但是那个时代最后却演化成了人人都吃不饱饭的时代。\par
在《共产党宣言》的原文中有这样一句话:“有人反驳说,私有制以消灭,一切活动就会停止,懒惰之风就会兴起”。而马克思恩格斯予以的反驳是,“这样说来,资产阶级社会早就应该因懒惰而灭亡了,因为在这个社会里是劳者不获,获者不劳的”。这句反驳事实上采用了反证法的思想,但在资产阶级所有制下,在那个年代,劳者不获这个论断,或许只正确了一半。作为被雇佣者的工人阶级,他们为资产阶级夜以继日的劳动,换来的只是微不足道的,连生活和家庭都难以维持的报酬,但报酬终究是有的。就像现在社会一样,许多人依然从事着下贱辛苦而又报酬少福利低的工作,他们并没有说自己工作的结果是一无所获就放弃找工作而选择去街上流浪。获者不劳这句话,从那个年代工人的视角看来,也确实是如此,但资本家未必是真的没有付出劳动,各种机器是他们发明创造的,他们也提出了各种经济学管理学理论,来使得生产利益尽可能的大等等,他们并非整日休闲度假,完全不劳而获。不过,他们付出的劳动可能相比真正的劳动者来说显得并不是那么多,而报酬却有些不合理的高。因此,我们可以总结说,工人阶级和资产阶级的矛盾,其实在于劳动付出与收获报酬间的不成比例。\par
理想情况下当然是像乌托邦那样,人人做出同样多的工作,获得同样的报酬。但这显得过于理论化了,首先,我们并不能衡量工作的多少。例如在高校里做研究的教授学者,或者再具体一点,一个研究宇宙学的科研小组,他们整天做各种理论计算和数值模拟,来研究我们生活的宇宙是从何而来的,又经历了怎样的演化过程来形成今天的宇宙大尺度结构,各种星系甚至我们人类自己。这个小组为了购买相关的计算资源,甚至说,大篇幅的草稿纸和笔,花费了许多科研经费,小组成员的月薪也比较可观。这时,校园里物业公司的保安听说了这个宇宙学小组的工作,愤愤不平,说这不就是杞人忧天么,知道宇宙怎么来的又怎么去,甚至都跟国家的生产水平没有半毛钱关系,他们的工作可以说根本没有价值。而我在这里看守大门还保护了学校师生的生命财产安全,那群做宇宙学的杞人凭什么工资还比我高呢?很多时候,只有自己才能明白,自己的工作究竟价值如何,工作量的强度又如何,但自我评判是非常主观的,而客观的标准事实上也不可能存在,所谓的客观评价只能是外行领导内行而已。因此,如何衡量两个人做的工作同样多,成了一个没有答案的问题。\par
更加现实的一个问题是,怎么保证没有人不偷懒,因为反正人人都能获得同样多的报酬。这个问题,在计划经济时代就显现出来了。用西方经济学的话讲就是,人会对激励作出反应,只有一个机制使得人能够劳有所获时,他才会去积极地生产工作,否则他宁愿整天在海滩度假。有人说,让国家监督每个人都同样努力地干活就好了,但问题是,我们没法去衡量每个人的工作,也就没法知道每个人是否都为社会付出了同等的努力;但他们获得的报酬却又是一样的,这不是又成了19世纪面临的情况,有人不劳而获,有人劳无所获么。\par
在今天我们有一个很好的例子来阐明这个问题:版权。目前中国对版权保护可以说是相当重视,国际上对待版权保护的态度则一向更加的严肃。按理,版权与生产资料紧密联系,甚至有时就是生产资料的一种。例如工业界对一些数值计算和数值模拟的软件十分依赖,使用这些软件可以减小很多在实验上的开销,但问题在于,这些软件一般都是由国外公司开发,对使用者收取不菲的费用。几年前,中国面对这一费用的态度便是,下载盗版。盗版软件的传播让大家都能够在互联网上平等地享有生产资料的获取权,但现在我们却发现,国家也开始主动地打击盗版。这背后的原因不难想到,盗版软件损害了软件开发者的利益,它们花费了自己的时间和心血开发的软件理应获得报酬,但盗版却让这一软件在互联网上免费地传播,这就打击了开发者的激情,宏观上来讲,也就削弱了整个国家对产业创新的激情。从短期来看,盗版这一使生产资料公有化的行为的确扩大了生产力,但长期而言,这种行为却又对生产力起到极大的阻碍作用。\par
简单来说,我坚持相信,消灭私有制还是会在一定程度上助长懒惰之风,除非社会的生产力已经发达到允许人们懒惰,而这却不是我们面对的实际情况。\par
但我们还是注意到,在《共产党宣言》写就的年代,工人们受压迫是明摆的事实,这一事实是反人性的,因此,改变依然需要做出。达到改变的第一步,还是如宣言中指出的一样,无产阶级必须登上政治舞台,从而利用政治力量改变自身处境。这一目标的达到,在我看来,和消灭资产阶级或消灭私有制并不冲突。在今天的西方社会,工人阶级就采取了组成工会、组成政党的形式维护自身的权益。或者说,在英国,资产阶级革命也是以非暴力的形式进行的,封建阶级也没有被完全推翻,但这并不妨碍英国成为第一次工业革命的起源地。从经济学角度讲,市场经济中人都是追求利益最大化的,在19世纪,工人阶级为了让自己拥有工作,宁愿用更低的工资去与其他工人竞争,导致了工人阶级整体工资和福利的低水平。这就相当于一个寡头竞争面临的问题,人人追求个人收益最大化带来的结果反而是损害同质群体中其他人的利益,最后导致整体利益的低下。一个改变的途径也很简单,就是联合起来,共同去争取权利,去对抗不公。\par
就像现在社会对966工作制的热烈讨论,互联网公司要求程序员整日整夜地加班工作,换来的反而是整体工作效率的低下,例如许多在强制要求966工作制的公司中的程序员采用所谓“摸鱼”的手段应对这一不合理的制度一样。更重要的是,这些程序员们选择将采用这一工作制的公司披露在程序员社区网站上,消除新人在单位选择上的信息不对称。这在某种程度上讲,与工人阶级的联合颇为相似。更进一步的设想,这些程序员是否有更加强有效的途径,例如法律层面,或采用劳动保障的方式去争取自己的利益呢?答案肯定是有的,这需要劳动者更加紧密地联合起来与各种既得利益者采取各种途径的斗争。放在19世纪,解决问题的思路依然相似,因此马克思和恩格斯明白,共产党人需要到处努力争取全世界的民主政党之间的团结和协调。\par
很大程度上,马克思恩格斯所期待的就是这样一个消除了不公正的社会。不公的消除在我看来并不一定要消灭这个消灭那个,而是,所有人能够坐到谈判桌前商讨;或者是,利用现代社会学管理学等社会科学的知识,去更加合理地制定政策,例如最低工资,社会福利保障等等。当然,不管达到这一目的的途径如何,第一步,依然是如马克思恩格斯在《共产党宣言》的最后所说的:“全世界无产者,联合起来”。

\end{document}
