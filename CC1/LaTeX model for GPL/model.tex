%A LaTeX model for SYSU General Phys. Lab. by Probfia Gao.
%用XeLaTeX编译
\documentclass[11pt,a4paper]{ctexart}

%在下面补全实验名,例如 实验BB3 光电效应实验。
\newcommand{\ExpeName}{实验CC1 热辐射的测量}

\usepackage{fancyhdr}
\usepackage{amsmath}
\usepackage{amssymb}
\usepackage{graphicx}
\usepackage[hmargin=1.25in,vmargin=1in]{geometry}
\usepackage{pdfpages}
\usepackage[colorlinks,
            linkcolor=red,
		 urlcolor=black]{hyperref}
\usepackage{cleveref}
\usepackage{float}

\crefname{equation}{}{}
\crefname{figure}{图}{图}
\crefname{footnote}{注释}{注释}
\crefname{table}{表}{表}

%\cpic{<尺寸>}{<文件名>}}用于生成居中的图片。
\newcommand{\cpic}[2]{
\begin{center}
\includegraphics[scale=#1]{#2}
\end{center}
}

%\cpicn{<尺寸>}{<文件名>}{<注释>}用于生成居中且带有注释的图片,其label为图片名。
\newcommand{\cpicn}[3]
{
\begin{figure}[H]
\cpic{#1}{#2}
\caption{#3\label{#2}}
\end{figure}
}

\newcommand{\beq}{\begin{equation}}
\newcommand{\eeq}{\end{equation}}
\newcommand{\bea}{\begin{equation}\begin{aligned}}
\newcommand{\eea}{\end{aligned}\end{equation}}

%输入单位和数学常数
%下面所有命令需在公式环境下使用
\newcommand{\e}{\mathrm{\ e}}   %自然常数e = \e
\newcommand{\im}{\mathrm{\ i}}   %虚数单位i = \im
\newcommand{\meter}{\mathrm{\ m}}      %单位/前缀 = \单位/前缀英文名
\newcommand{\newton}{\mathrm{\ N}}  
\newcommand{\joule}{\mathrm{\ J}}
\newcommand{\second}{\mathrm{\ s}}
\newcommand{\gram}{\mathrm{\ g}}
\newcommand{\ampere}{\mathrm{\ A}}
\newcommand{\kilogram}{\mathrm{\ kg}}
\newcommand{\kelvin}{\mathrm{\ K}}
\newcommand{\mole}{\mathrm{\ mol}}
\newcommand{\volt}{\mathrm{\ V}}
\newcommand{\degreeC}{\ ^\circ \mathrm{C}}  %摄氏度符号 = \degreeC


\newcommand{\emptyline}{\par \ \\}


\pagestyle{fancy}

\fancyhead[L]{\footnotesize{中山大学物理与天文学院基础物理实验}}
\fancyhead[R]{\footnotesize{\ExpeName}}
\fancyfoot[C]{\thepage}

\begin{document}
%第一页
\cpic{0.255}{e1}%学生信息和计分表格
\begin{center}
\LARGE\textbf{{\ExpeName}}
\end{center}
\large{【实验报告注意事项】}
\begin{enumerate}
 \item 实验报告由三部分组成:
 \begin{enumerate}
  \item[1)]预习报告:(提前一周)认真研读\textbf{\uline{实验讲义}},弄清实验原理;实验所需的仪器设备、用具及其使用(强烈建议到实验室预习),完成讲义中的预习思考题;了解实验需要测量的物理量,并根据要求提前准备实验记录表格(由学生自己在实验前设计好,可以打印)。预习成绩低于10分(共20分)者不能做实验。
  \item[2)]实验记录:认真、客观记录实验条件、实验过程中的现象以及数据。实验记录请用珠笔或者钢笔书写并签名({\color{red}用铅笔记录的被认为无效})。{\color{red}保持原始记录,包括写错删除部分,如因误记需要修改记录,必须按规范修改。}(不得输入电脑打印,但可扫描手记后打印扫描件);离开前请实验教师检查记录并签名。
  \item[3)]分析讨论:处理实验原始数据(学习仪器使用类型的实验除外),对数据的可靠性和合理性进行分析;按规范呈现数据和结果(图、表),包括数据、图表按顺序编号及其引用;分析物理现象(含回答实验思考题,写出问题思考过程,必要时按规范引用数据);最后得出结论。
 \end{enumerate}
 \textbf{实验报告}就是预习报告、实验记录、和数据处理与分析合起来,加上本页封面。
 \item 每次完成实验后的一周内交\textbf{实验报告}。
 \item 除实验记录外,实验报告其他部分建议双面打印。
\end{enumerate}
\ 
\\
\ 

\begin{flushright}                                                           %模板作者
\tiny{
A \LaTeX \ model for General Phys. Lab., SPA, SYSU by {\em \href{https://www.weibo.com/3532532974/profile?rightmod=1&wvr=6&mod=personinfo&is_all=1}{Probfia} Gao.}\\ Adopted from the \href{http://lovephysics.sysu.edu.cn/lib/exe/fetch.php?media=courses:secondlevelzhuhai:report.docx}{original MS Word model} on \href{http://lovephysics.sysu.edu.cn}{Lovephysics}.\\ You can view it on \href{https://github.com/Probfia/SYSU_GPL_C}{Github}.}
\end{flushright}

\newpage%预习报告
\begin{center}
\LARGE{\textbf{\ExpeName}}
\end{center}
\textbf{【实验目的】}
\begin{enumerate}
 \item[1.] 认识普遍存在的热辐射现象及其本质——一种能量转换与传递的形式;
 \item[2.] 通过相对低温的热辐射实验认识影响热辐射强度的各种因素及其与热辐射强度的定量关系;包括:
\begin{enumerate}
\item[a)] 辐射体表面温度;
\item[b)] 辐射距离\footnote{测试点与辐射体表面距离};(面源辐射修正)
\item[c)] 表面的发射系数(拓展);
\item[d)] 验证上述因素与辐射强度的定量关系是否符合黑体辐射定律。
\end{enumerate}
 \item[3.] 了解热辐射传感器(SMTIR9902)原理和结构、使用(含校正)方法;
 \item[4.] 学习应用LabView管理由具有NI通信协议的非NI专业仪器(数字多用表)、设备(程控电源)构成的实验系统。
\end{enumerate}
\textbf{【仪器用具】}
%将讲义中的表格截图保存为t1在该文件夹下后删去下一行之前的%符号,合理调整scale参数。
\cpic{0.3}{t1}
%或者自己去 https://www.tablesgenerator.com/ 做一个表。
\textbf{【原理概述】}\par
该实验测量热辐射强度与温度、距离间的关系。\par
根据热力学可以证明,黑体的热辐射功率$P$与温度的四次方成正比
\beq
P = \sigma T^4
\eeq
而根据辐射强度(单位面积上的辐射功率)守恒原则,预期辐射强度与距离的平方呈反比关系
\beq
I(s) \propto \frac{1}{r^2}
\eeq
\par
本实验利用一个温度可控的辐射体和一个由热电偶构成的辐射传感器探究上述关系的正确性。
\emptyline
\textbf{【实验前思考题】}
\begin{enumerate}
 \item[1.]\textbf{人体热辐射会对传感器计数产生影响(自己可验证),如何消除这种影响?}\par
坐远点,坐低点,趴到地下去。
 \item[2.]\textbf{辐射体的加热功率与辐射体温度之间呈何关系?与辐射传感器的信号值之间呈何关系?为什么?}\par
敬待实验后解答。
 \item[3.]\textbf{为何当热辐射传感器太靠近辐射表面时,所没得的辐射强度偏离距离平方反比规律?}\par
平方反比规律的成立很大程度上依赖辐射在一定立体角$\varOmega$内呈各向同性的假设,而在辐射体表面,由于表面本身形状非弧形,使得辐射明显呈各向异性。容易估计出,辐射各向同性的假设需要在距离$s \gtrsim l$范围内成立,其中$l$为辐射体的线度。
 \item[4.]\textbf{能否将SMTIR9902应用于非接触温度测量?}\par
\end{enumerate}

\newpage%实验记录
\cpic{0.255}{e2}%学生信息表格
\begin{center}
\LARGE{\textbf{\ExpeName}}
\end{center}
\textbf{【实验内容、步骤、结果】}
\\
1.\textbf{实验系统搭建}\par
按讲义要求熟悉并搭建实验系统。
\emptyline
2.\textbf{测量物体的辐射面温度对物体辐射强度大小的影响}\par
选择几组不同的辐射距离值,探究辐射强度与辐射表面温度之间的关系。表格不必全部用完。
\begin{table}[H]
\centering
\caption{距离$s = $\uline{\hspace{2cm}}时,辐射强度与辐射表面温度之间的关系}
\label{t1}
\begin{tabular}{|c|p{12mm}|p{12mm}|p{12mm}|p{12mm}|p{12mm}|p{12mm}|p{12mm}|}
\hline
温度$t/\degreeC$ &  &  &  &  &  &  &  \\ \hline
辐射强度$P/\mathrm{V}$ &  &  &  &  &  &  &  \\ \hline
电源输出功率/W &  &  &  &  &  &  &  \\ \hline
\end{tabular}
\end{table}
\begin{table}[H]
\centering
\caption{距离$s = $\uline{\hspace{2cm}}时,辐射强度与辐射表面温度之间的关系}
\label{t2}
\begin{tabular}{|c|p{12mm}|p{12mm}|p{12mm}|p{12mm}|p{12mm}|p{12mm}|p{12mm}|}
\hline
温度$t/\degreeC$ &  &  &  &  &  &  &  \\ \hline
辐射强度$P/\mathrm{V}$ &  &  &  &  &  &  &  \\ \hline
电源输出功率/W &  &  &  &  &  &  &  \\ \hline
\end{tabular}
\end{table}
\begin{table}[H]
\centering
\caption{距离$s = $\uline{\hspace{2cm}}时,辐射强度与辐射表面温度之间的关系}
\label{t3}
\begin{tabular}{|c|p{12mm}|p{12mm}|p{12mm}|p{12mm}|p{12mm}|p{12mm}|p{12mm}|}
\hline
温度$t/\degreeC$ &  &  &  &  &  &  &  \\ \hline
辐射强度$P/\mathrm{V}$ &  &  &  &  &  &  &  \\ \hline
电源输出功率/W &  &  &  &  &  &  &  \\ \hline
\end{tabular}
\end{table}
\begin{table}[H]
\centering
\caption{距离$s = $\uline{\hspace{2cm}}时,辐射强度与辐射表面温度之间的关系}
\label{t4}
\begin{tabular}{|c|p{12mm}|p{12mm}|p{12mm}|p{12mm}|p{12mm}|p{12mm}|p{12mm}|}
\hline
温度$t/\degreeC$ &  &  &  &  &  &  &  \\ \hline
辐射强度$P/\mathrm{V}$ &  &  &  &  &  &  &  \\ \hline
电源输出功率/W &  &  &  &  &  &  &  \\ \hline
\end{tabular}
\end{table}
\emptyline
3.\textbf{测量物体在不同辐射距离$s$的辐射强度$P$,通过拟合给出$P-s$之间的关系}\par
将温度控制在某个固定值,移动红外传感器的位置,测量辐射强度,填入下表。
\begin{table}[H]
\centering
\caption{温度$t = $\uline{\hspace{2cm}}$\degreeC$时,不同辐射距离的辐射强度}
\label{t5}
\begin{tabular}{|c||p{12mm}|p{12mm}|p{12mm}|p{12mm}|p{12mm}|p{12mm}|p{12mm}|}
\hline
距离$s/\mathrm{cm}$ &  &  &  &  &  &  &  \\ \hline
辐射强度$P/\mathrm{mV}$ &  &  &  &  &  &  &  \\ \hline
\end{tabular}
\end{table}
\emptyline
4.\textbf{测量不同物体表面的发射系数}\par
控制温度为$t = 60\ \degreeC$,将传感器尽可能移近辐射体,测量不同表面的辐射强度,数据填入下表。
\begin{table}[H]
\centering
\caption{测量不同物体表面的辐射强度}
\label{t6}
\begin{tabular}{|c|c|p{12mm}|p{12mm}|p{12mm}|}
\hline
所用距离 & 辐射面 & 黑面 & 粗糙面 & 光面 \\ \cline{2-5} 
$s = $\uline{\hspace{15mm}}cm & 辐射强度$P/\mathrm{V}$ &  &  &  \\ \hline
\end{tabular}
\end{table}
\emptyline
\textbf{【实验过程中遇到问题记录】}

%生成最终报告时将上面内容全部删除或注释(用\iffalse \fi),将扫描得到的实验报告保存为Record.pdf在LaTeX model for GPL下,将下行命令的注释号删去。注意根据实际页数调整pages参数。
%\includepdf[pages=1-3]{Record}

\newpage%分析与讨论
\cpic{0.255}{e3}%学生信息表格
\begin{center}
\LARGE\textbf{{\ExpeName}}
\end{center}
\textbf{【分析与讨论】}\par
(Content)
\emptyline
\textbf{【实验思考题】}\par
(Content)

\end{document}