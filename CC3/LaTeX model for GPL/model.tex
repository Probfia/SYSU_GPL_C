%A LaTeX model for SYSU General Phys. Lab. by Probfia Gao.
%用XeLaTeX编译
\documentclass[11pt,a4paper]{ctexart}

%在下面补全实验名,例如 实验BB3 光电效应实验。
\newcommand{\ExpeName}{实验CC3 微振动基本实验(双光栅测量微弱振动实验)}

\usepackage{fancyhdr}
\usepackage{amsmath}
\usepackage{graphicx}
\usepackage[hmargin=1.25in,vmargin=1in]{geometry}
\usepackage{pdfpages}
\usepackage[colorlinks,
            linkcolor=red,
		 urlcolor=black]{hyperref}
\usepackage{cleveref}

\crefname{equation}{}{}
\crefname{figure}{图}{图}
\crefname{footnote}{注释}{注释}

%\cpic{<尺寸>}{<文件名>}}用于生成居中的图片。
\newcommand{\cpic}[2]{
\begin{center}
\includegraphics[scale=#1]{#2}
\end{center}
}

%\cpicn{<尺寸>}{<文件名>}{<注释>}用于生成居中且带有注释的图片,其label为图片名。
\newcommand{\cpicn}[3]
{
\begin{figure}[h!]
\cpic{#1}{#2}
\caption{#3\label{#2}}
\end{figure}
}

\newcommand{\beq}{\begin{equation}}
\newcommand{\eeq}{\end{equation}}
\newcommand{\bea}{\begin{equation}\begin{aligned}}
\newcommand{\eea}{\end{aligned}\end{equation}}

%输入单位和数学常数
%下面所有命令需在公式环境下使用
\newcommand{\e}{\mathrm{\ e}}   %自然常数e = \e
\newcommand{\im}{\mathrm{\ i}}   %虚数单位i = \im
\newcommand{\meter}{\mathrm{\ m}}      %单位/前缀 = \单位/前缀英文名
\newcommand{\newton}{\mathrm{\ N}}  
\newcommand{\joule}{\mathrm{\ J}}
\newcommand{\second}{\mathrm{\ s}}
\newcommand{\gram}{\mathrm{\ g}}
\newcommand{\ampere}{\mathrm{\ A}}
\newcommand{\kilogram}{\mathrm{\ kg}}
\newcommand{\kelvin}{\mathrm{\ K}}
\newcommand{\mole}{\mathrm{\ mol}}
\newcommand{\volt}{\mathrm{\ V}}
\newcommand{\degreeC}{\ ^\circ \mathrm{C}}  %摄氏度符号 = \degreeC



\pagestyle{fancy}

\fancyhead[L]{\footnotesize{中山大学物理与天文学院基础物理实验}}
\fancyhead[R]{\footnotesize{\ExpeName}}
\fancyfoot[C]{\thepage}

\begin{document}
%第一页
\cpic{0.255}{e1}%学生信息和计分表格
\begin{center}
\LARGE\textbf{{\ExpeName}}
\end{center}
\large{【实验报告注意事项】}
\begin{enumerate}
 \item 实验报告由三部分组成:
 \begin{enumerate}
  \item[1)]预习报告:(提前一周)认真研读\textbf{\uline{实验讲义}},弄清实验原理;实验所需的仪器设备、用具及其使用(强烈建议到实验室预习),完成讲义中的预习思考题;了解实验需要测量的物理量,并根据要求提前准备实验记录表格(由学生自己在实验前设计好,可以打印)。预习成绩低于10分(共20分)者不能做实验。
  \item[2)]实验记录:认真、客观记录实验条件、实验过程中的现象以及数据。实验记录请用珠笔或者钢笔书写并签名({\color{red}用铅笔记录的被认为无效})。{\color{red}保持原始记录,包括写错删除部分,如因误记需要修改记录,必须按规范修改。}(不得输入电脑打印,但可扫描手记后打印扫描件);离开前请实验教师检查记录并签名。
  \item[3)]分析讨论:处理实验原始数据(学习仪器使用类型的实验除外),对数据的可靠性和合理性进行分析;按规范呈现数据和结果(图、表),包括数据、图表按顺序编号及其引用;分析物理现象(含回答实验思考题,写出问题思考过程,必要时按规范引用数据);最后得出结论。
 \end{enumerate}
 \textbf{实验报告}就是预习报告、实验记录、和数据处理与分析合起来,加上本页封面。
 \item 每次完成实验后的一周内交\textbf{实验报告}。
 \item 除实验记录外,实验报告其他部分建议双面打印。
\end{enumerate}
\ 
\\
\ 

\begin{flushright}                                                           %模板作者
\tiny{
A \LaTeX \ model for General Phys. Lab., SPA, SYSU by {\em \href{https://www.weibo.com/3532532974/profile?rightmod=1&wvr=6&mod=personinfo&is_all=1}{Probfia} Gao.}\\ Adopted from the \href{http://lovephysics.sysu.edu.cn/lib/exe/fetch.php?media=courses:secondlevelzhuhai:report.docx}{original MS Word model} on \href{http://lovephysics.sysu.edu.cn}{Lovephysics}.\\ You can view it on \href{https://github.com/Probfia/SYSU_GPL_C}{Github}.}
\end{flushright}

\newpage%预习报告
\begin{center}
\LARGE{\textbf{\ExpeName}}
\end{center}
\textbf{【实验目的】}
\begin{enumerate}
 \item[1.] 了解利用光的多普勒频移形成光拍的原理并用于测量光拍拍频。
 \item[2.] 学会使用精确测量微弱振动位移的一种方法。
 \item[3.] 应用双光栅微弱振动实验仪测量音叉振动的微振幅
\end{enumerate}
\textbf{【仪器用具】}
%将讲义中的表格截图保存为t1在该文件夹下后删去下一行之前的%符号,合理调整scale参数。
%cpic{0.3}{t1}
%或者自己去 https://www.tablesgenerator.com/ 做一个表。
\begin{table}[h!]
\centering
\begin{tabular}{|c|c|c|}
\hline
仪器名称 & 数量 & 仪器参数 \\ \hline
半导体激光器 & 1 & $\lambda=650\mathrm{\ nm}$,功率 $2-5\mathrm{\ mW}$ \\ \hline
音叉 & 1 & 频率500 Hz左右 \\ \hline
示波器等 & 1 & 位移量分辨率:$5\mathrm{\ \mu m}$ \\ \hline
\end{tabular}
\end{table}
\textbf{【原理概述】}\par
利用光的多普勒频移形成拍,并以此测量微振动的位移。
\par
光栅衍射的极大值位置满足
\beq
k\lambda = d\sin \theta
\eeq
但当光从$x$方向入射,而光栅在$y$方向上运动的时候,它将在$t$时间内带来额外的相位差
\bea
\Delta \phi &= \frac{2\pi}{\lambda} vt\sin \theta \\
&= \frac{2\pi}{\lambda}vt\frac{k\lambda}{d} \\
&= \frac{k}{\omega_d} t
\eea
其中$\omega_d = 2\pi \cfrac{v}{d}$。移动的光栅带来光频率的变化,这就是光的多普勒频移。
\par
如果有两个光栅,一个静止而一个以$v$的速度移动,那么,它们形成的衍射光将叠加在一起。忽略它们原有的相位差,两个光束叠加后的强度为
\bea
I &= (E_1(t) +E_2(t))^2 \\
&= E_1^2 \cos^2 \omega_0 t + E_2^2 \cos^2 (\omega_0 + \omega_d)t + 2E_1 E_2 \cos \omega_0 t \cos (\omega_0 + \omega_d)t \\
&= E_1^2 \cos^2 \omega_0 t + E_2^2 \cos^2 (\omega_0 + \omega_d)t + E_1 E_2 \cos \omega_d t + E_1 E_2 \cos (2\omega_0 - \omega_d)t
\eea
这里$\omega_0 \gg \omega_d$,且仪器无法对$\sim \omega_0$的光强变化做出相应,故示波器上将只显示以$\omega_d$为频率振动的拍频,它在示波器上显示的频率为
\bea
f_d &= \frac{\omega_d}{2\pi} \\
&= \frac{2\pi v/d}{2\pi} \\
&= \frac{v}{d}
\eea
其中$d = 0.01\mathrm{\ mm}$为实验中的光栅常数,故以拍频计算运动速度的公式就是
\beq
v = 0.01\frac{f}{\mathrm{Hz}} \mathrm{\ mm/s}
\eeq
如果$v$随时间变化,$f$也将随时间变化,再进一步地,如果$v$以正弦(余弦)形式变化,那么,简谐运动的振幅就是
\beq
A = \frac{1}{2} \int_{t=0}^{\frac{T}{2}} v(t) dt = 0.005 \int_{t=0}^{\frac{T}{2}} f(t)dt\mathrm{\ mm} 
\eeq
积分就是拍频的个数,于是我们就得到了微振动的振幅。

\newpage%实验记录
\cpic{0.255}{e2}%学生信息表格
\begin{center}
\LARGE{\textbf{\ExpeName}}
\end{center}
\textbf{【实验内容、步骤、结果】}
\newline
1.\textbf{熟悉各仪器的使用方法}\par
按照讲义上的要求和知道,熟悉装置的组成,掌握各仪器的使用方法。
\\
\ 
\\
2.\textbf{光路和音叉的调整}\par
调整光路,使得各光学仪器共线;调整使得两光栅尽可能平行。\par
按讲义要求调整驱动力频率使得音叉发生谐振,此时能够在示波器中看到15个左右的波形,此时的实验相关数据如下
\begin{table}[h!]
\centering
\caption{音叉谐振的振幅}
\begin{tabular}{|c|p{20mm}|}
\hline
频率/Hz &  \\ \hline
$T/2$时间内的波形数 &  \\ \hline
音叉振幅/mm &  \\ \hline
\end{tabular}
\end{table}
\newline
\ 
\\
3.\textbf{音叉谐振曲线的测量}\par
在谐振点附近调节驱动力频率,测得频率与振幅的关系如下(表格不必全部用完)。
\begin{table}[h!]
\centering
\caption{音叉谐振曲线的测量}
\begin{tabular}{|c|p{9mm}|p{9mm}|p{9mm}|p{9mm}|p{9mm}|p{9mm}|p{9mm}|p{9mm}|}
\hline
频率/Hz &  &  &  &  &  &  &  &  \\ \hline
$T/2$时间内的波形数 &  &  &  &  &  &  &  &  \\ \hline
音叉振幅/mm &  &  &  &  &  &  &  &  \\ \hline
\end{tabular}
\end{table}
\begin{table}[h!]
\centering
\caption{音叉谐振曲线的测量(续表)}
\begin{tabular}{|c|p{9mm}|p{9mm}|p{9mm}|p{9mm}|p{9mm}|p{9mm}|p{9mm}|p{9mm}|}
\hline
频率/Hz &  &  &  &  &  &  &  &  \\ \hline
$T/2$时间内的波形数 &  &  &  &  &  &  &  &  \\ \hline
音叉振幅/mm &  &  &  &  &  &  &  &  \\ \hline
\end{tabular}
\end{table}
\\
\ 
\\
4.\textbf{音叉质量对谐振曲线的影响}\par
将软管插入音叉,再次测量驱动力频率与振幅的关系如下(表格不必全部用完)。
\begin{table}[h!]
\centering
\caption{软管插入在\uline{\hspace{45mm}}时,音叉谐振曲线的测量}
\begin{tabular}{|c|p{9mm}|p{9mm}|p{9mm}|p{9mm}|p{9mm}|p{9mm}|p{9mm}|p{9mm}|}
\hline
频率/Hz &  &  &  &  &  &  &  &  \\ \hline
$T/2$时间内的波形数 &  &  &  &  &  &  &  &  \\ \hline
音叉振幅/mm &  &  &  &  &  &  &  &  \\ \hline
\end{tabular}
\end{table}
\begin{table}[h!]
\centering
\caption{软管插入在\uline{\hspace{45mm}}时,音叉谐振曲线的测量}
\begin{tabular}{|c|p{9mm}|p{9mm}|p{9mm}|p{9mm}|p{9mm}|p{9mm}|p{9mm}|p{9mm}|}
\hline
频率/Hz &  &  &  &  &  &  &  &  \\ \hline
$T/2$时间内的波形数 &  &  &  &  &  &  &  &  \\ \hline
音叉振幅/mm &  &  &  &  &  &  &  &  \\ \hline
\end{tabular}
\end{table}
\begin{table}[h!]
\centering
\caption{软管插入在\uline{\hspace{45mm}}时,音叉谐振曲线的测量}
\begin{tabular}{|c|p{9mm}|p{9mm}|p{9mm}|p{9mm}|p{9mm}|p{9mm}|p{9mm}|p{9mm}|}
\hline
频率/Hz &  &  &  &  &  &  &  &  \\ \hline
$T/2$时间内的波形数 &  &  &  &  &  &  &  &  \\ \hline
音叉振幅/mm &  &  &  &  &  &  &  &  \\ \hline
\end{tabular}
\end{table}
\\
\textbf{【实验过程中遇到问题记录】}

%生成最终报告时将上面内容全部删除或注释(用\iffalse \fi),将扫描得到的实验报告保存为Record.pdf在LaTeX model for GPL下,将下行命令的注释号删去。注意根据实际页数调整pages参数。
%\includepdf[pages=1-3]{Record}

\newpage%分析与讨论
\cpic{0.255}{e3}%学生信息表格
\begin{center}
\LARGE\textbf{{\ExpeName}}
\end{center}
\textbf{【分析与讨论】}\par
(Content)
\newline
\textbf{【实验思考题】}\par
(Content)

\end{document}