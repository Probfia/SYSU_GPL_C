%A LaTeX model for SYSU General Phys. Lab. by Probfia Gao.
%用XeLaTeX编译
\documentclass[11pt,a4paper]{ctexart}

%在下面补全实验名,例如 实验BB3 光电效应实验。
\newcommand{\ExpeName}{实验CA2 弗兰克-赫兹实验:原子定态能级的观测}

\usepackage{fancyhdr}
\usepackage{amsmath}
\usepackage{graphicx}
\usepackage[hmargin=1.25in,vmargin=1in]{geometry}
\usepackage{pdfpages}
\usepackage[colorlinks,
            linkcolor=red,
		 urlcolor=black]{hyperref}
\usepackage{cleveref}

\crefname{equation}{}{}
\crefname{figure}{图}{图}
\crefname{footnote}{注释}{注释}

%\cpic{<尺寸>}{<文件名>}}用于生成居中的图片。
\newcommand{\cpic}[2]{
\begin{center}
\includegraphics[scale=#1]{#2}
\end{center}
}

%\cpicn{<尺寸>}{<文件名>}{<注释>}用于生成居中且带有注释的图片,其label为图片名。
\newcommand{\cpicn}[3]
{
\begin{figure}[h!]
\cpic{#1}{#2}
\caption{#3\label{#2}}
\end{figure}
}

\newcommand{\beq}{\begin{equation}}
\newcommand{\eeq}{\end{equation}}
\newcommand{\bea}{\begin{equation}\begin{aligned}}
\newcommand{\eea}{\end{aligned}\end{equation}}

%输入单位和数学常数
%下面所有命令需在公式环境下使用
\newcommand{\e}{\mathrm{e}}   %自然常数e = \e
\newcommand{\im}{\mathrm{i}}   %虚数单位i = \im
\newcommand{\meter}{\mathrm{m}}      %单位/前缀 = \单位/前缀英文名
\newcommand{\newton}{\mathrm{N}}  
\newcommand{\joule}{\mathrm{J}}
\newcommand{\second}{\mathrm{s}}
\newcommand{\gram}{\mathrm{g}}
\newcommand{\ampere}{\mathrm{A}}
\newcommand{\kilo}{\mathrm{k}}
\newcommand{\milli}{\mathrm{m}}
\newcommand{\kelvin}{\mathrm{K}}
\newcommand{\mole}{\mathrm{mol}}
\newcommand{\volt}{\mathrm{V}}
\newcommand{\nano}{\mathrm{n}}
\newcommand{\degreeC}{^\circ \mathrm{C}}  %摄氏度符号 = \degreeC



\pagestyle{fancy}

\fancyhead[L]{\footnotesize{中山大学物理与天文学院基础物理实验}}
\fancyhead[R]{\footnotesize{\ExpeName}}
\fancyfoot[C]{\thepage}

\begin{document}
%第一页
\cpic{0.255}{e1}%学生信息和计分表格
\begin{center}
\LARGE\textbf{{\ExpeName}}
\end{center}
\large{【实验报告注意事项】}
\begin{enumerate}
 \item 实验报告由三部分组成:
 \begin{enumerate}
  \item[1)]预习报告:(提前一周)认真研读\textbf{\uline{实验讲义}},弄清实验原理;实验所需的仪器设备、用具及其使用(强烈建议到实验室预习),完成讲义中的预习思考题;了解实验需要测量的物理量,并根据要求提前准备实验记录表格(由学生自己在实验前设计好,可以打印)。预习成绩低于10分(共20分)者不能做实验。
  \item[2)]实验记录:认真、客观记录实验条件、实验过程中的现象以及数据。实验记录请用珠笔或者钢笔书写并签名({\color{red}用铅笔记录的被认为无效})。{\color{red}保持原始记录,包括写错删除部分,如因误记需要修改记录,必须按规范修改。}(不得输入电脑打印,但可扫描手记后打印扫描件);离开前请实验教师检查记录并签名。
  \item[3)]分析讨论:处理实验原始数据(学习仪器使用类型的实验除外),对数据的可靠性和合理性进行分析;按规范呈现数据和结果(图、表),包括数据、图表按顺序编号及其引用;分析物理现象(含回答实验思考题,写出问题思考过程,必要时按规范引用数据);最后得出结论。
 \end{enumerate}
 \textbf{实验报告}就是预习报告、实验记录、和数据处理与分析合起来,加上本页封面。
 \item 每次完成实验后的一周内交\textbf{实验报告}。
 \item 除实验记录外,实验报告其他部分建议双面打印。
\end{enumerate}
\ 
\\
\ 

\begin{flushright}                                                           %模板作者
\tiny{
A \LaTeX \ model for General Phys. Lab., SPA, SYSU by \em{\href{https://www.weibo.com/3532532974/profile?rightmod=1&wvr=6&mod=personinfo&is_all=1}{Probfia} Gao.}
}
\end{flushright}

\newpage%预习报告
\begin{center}
\LARGE{\textbf{\ExpeName}}
\end{center}
\textbf{【实验目的】}
\begin{enumerate}
 \item[1.]学习弗兰克-赫兹实验仪的使用方法。
 \item[2.]测量氩原子的电流-加速电压关系曲线,计算第一激发电位
\end{enumerate}
\textbf{【仪器用具】}\par
%将讲义中的表格截图保存为t1在该文件夹下后删去下一行之前的%符号,合理调整scale参数。
%cpic{0.3}{t1}
%或者自己去 https://www.tablesgenerator.com/ 做一个表。
弗兰克-赫兹实验仪(世纪中科ZKY-FH-2或博洋光电BEX-8502),氩管,汞管及控温装置,数字示波器。  \\ 
\textbf{【原理概述】}\par
该实验测量氩原子的能级。当一个能量恰为两能级间能量差的电子与原子碰撞时,原子将吸收这个电子的能量并跃迁到高能级,此时,电子的能量将被吸收。
\par
实验的装置图如\cref{princ1}。
\cpicn{0.5}{princ1}{实验装置原理图}
$U_{G_1 K}$用于加速发散的电子,使之具有能量$eU_{G_1 K}$,而$U_{AG_1}$使得电子减速,阻止其到底极板$A$。但只要
\beq
U_{G_1 K} > U_{AG_1}
\eeq
电子就可能到达极板$A$,并在微电流计上显示出电流,除非电子在$G_2$区域附近与原子发生碰撞而损失能量,这一碰撞发生的条件就是
\beq \label{collcond}
eU_{G_2 K} = E_n - E_0
\eeq
其中$E_n$是氩原子的第$n$能级能量。这时电流计检测到的电流就会发生突变,体现为一个极小值。在$I - U_{G_2 K}$图像上找出各极小值点$V_n$,就可以带入\cref{collcond}计算出氩原子各激发态与基态间的能量差。
\\
\ 
\\
\textbf{【实验前思考题】}
\begin{enumerate}
 \item[1.]\textbf{何为 F-H 的最佳工作点,实验中如何确定?}\par
F-H的最佳工作点是使得电流大小适当,实验曲线清晰光滑的灯丝电压$U_F$、第一级电压$U_{G_1 K}$和拒斥电压$U_{AG_2}$的那组值。可以通过前面的提到的判断依据确定最佳工作点的大致值\footnote{王杰, 司嵘嵘. 确定弗兰克-赫兹实验最佳工作参数的方法改进[J]. 大学物理实验, 2018, 第31卷(5):87-91.\label{bestpt}},根据\cref{bestpt},$U_F= 3.1\volt$,$U_{G_1 K} = 1.4\volt$和$U_{AG_2} = 6.0\volt$的实验结果较为理想。
 \item[2.]\textbf{为什么其$I(V)$曲线不是断崖式的下落?为什么电流不会下降为零?}\par
由于灯丝发射的电压具有一定的初动能,且在一定范围内分布,这使得电子的能量有微小的差别,不同的电子在不同电压值发生碰撞,但总体而言曲线峰值依然在理论值附近。由于不是所有电子都能与氦原子发生碰撞,电流不可能骤降为零。
\end{enumerate}

\newpage%实验记录
\cpic{0.255}{e2}%学生信息表格
\begin{center}
\LARGE{\textbf{\ExpeName}}
\end{center}
\textbf{【实验内容、步骤、结果】}\par
该实验自动采集数据。按照步骤连线、初始化后,输入参数即可开始自动采集数据。重复测量,得到氩原子的能级关系。
\par
改变灯丝电压,探究灯丝电压设置对曲线的影响。
\newline 
\ 
\\
\textbf{【实验过程中遇到问题记录】}

%生成最终报告时将上面内容全部删除或注释(用\iffalse \fi),将扫描得到的实验报告保存为Record.pdf在LaTeX model for GPL下,将下行命令的注释号删去。注意根据实际页数调整pages参数。
%\includepdf[pages=1-3]{Record}

\newpage%分析与讨论
\cpic{0.255}{e3}%学生信息表格
\begin{center}
\LARGE\textbf{{\ExpeName}}
\end{center}
\textbf{【分析与讨论】}\par
(Content)
\newline
\textbf{【实验思考题】}\par
(Content)

\end{document}