%A LaTeX model for SYSU General Phys. Lab. by Probfia Gao.
%用XeLaTeX编译
\documentclass[11pt,a4paper]{ctexart}

%在下面补全实验名,例如 实验BB3 光电效应实验。
\newcommand{\ExpeName}{设计性实验-基于微小振动测量平台}

\usepackage{fancyhdr}
\usepackage{amsmath}
\usepackage{amssymb}
\usepackage{graphicx}
\usepackage[hmargin=1.25in,vmargin=1in]{geometry}
\usepackage{pdfpages}
\usepackage[colorlinks,
            linkcolor=red,
		 urlcolor=black]{hyperref}
\usepackage{cleveref}
\usepackage{float}

\crefname{equation}{}{}
\crefname{figure}{图}{图}
\crefname{footnote}{注释}{注释}
\crefname{table}{表}{表}

%\cpic{<尺寸>}{<文件名>}}用于生成居中的图片。
\newcommand{\cpic}[2]{
\begin{center}
\includegraphics[scale=#1]{#2}
\end{center}
}

%\cpicn{<尺寸>}{<文件名>}{<注释>}用于生成居中且带有注释的图片,其label为图片名。
\newcommand{\cpicn}[3]
{
\begin{figure}[H]
\cpic{#1}{#2}
\caption{#3\label{#2}}
\end{figure}
}

\newcommand{\beq}{\begin{equation}}
\newcommand{\eeq}{\end{equation}}
\newcommand{\bea}{\begin{equation}\begin{aligned}}
\newcommand{\eea}{\end{aligned}\end{equation}}

%输入单位和数学常数
%下面所有命令需在公式环境下使用
\newcommand{\e}{\mathrm{\ e}}   %自然常数e = \e
\newcommand{\im}{\mathrm{\ i}}   %虚数单位i = \im
\newcommand{\meter}{\mathrm{\ m}}      %单位/前缀 = \单位/前缀英文名
\newcommand{\newton}{\mathrm{\ N}}  
\newcommand{\joule}{\mathrm{\ J}}
\newcommand{\second}{\mathrm{\ s}}
\newcommand{\gram}{\mathrm{\ g}}
\newcommand{\ampere}{\mathrm{\ A}}
\newcommand{\kilogram}{\mathrm{\ kg}}
\newcommand{\kelvin}{\mathrm{\ K}}
\newcommand{\mole}{\mathrm{\ mol}}
\newcommand{\volt}{\mathrm{\ V}}
\newcommand{\degreeC}{\ ^\circ \mathrm{C}}  %摄氏度符号 = \degreeC


\newcommand{\emptyline}{\\ \ \\}


\pagestyle{fancy}

\fancyhead[L]{\footnotesize{中山大学物理与天文学院基础物理实验}}
\fancyhead[R]{\footnotesize{\ExpeName}}
\fancyfoot[C]{\thepage}

\begin{document}
%第一页
\cpic{0.34}{e1}%学生信息和计分表格
\begin{center}
\LARGE\textbf{{\ExpeName}}
\end{center}
\large{【实验报告注意事项】}
\begin{enumerate}
 \item 实验报告由三部分组成:
 \begin{enumerate}
  \item[1)]预习报告:(提前一周)认真研读\textbf{\uline{实验讲义}},弄清实验原理;实验所需的仪器设备、用具及其使用(强烈建议到实验室预习),完成讲义中的预习思考题;了解实验需要测量的物理量,并根据要求提前准备实验记录表格(由学生自己在实验前设计好,可以打印)。预习成绩低于10分(共20分)者不能做实验。
  \item[2)]实验记录:认真、客观记录实验条件、实验过程中的现象以及数据。实验记录请用珠笔或者钢笔书写并签名({\color{red}用铅笔记录的被认为无效})。{\color{red}保持原始记录,包括写错删除部分,如因误记需要修改记录,必须按规范修改。}(不得输入电脑打印,但可扫描手记后打印扫描件);离开前请实验教师检查记录并签名。
  \item[3)]分析讨论:处理实验原始数据(学习仪器使用类型的实验除外),对数据的可靠性和合理性进行分析;按规范呈现数据和结果(图、表),包括数据、图表按顺序编号及其引用;分析物理现象(含回答实验思考题,写出问题思考过程,必要时按规范引用数据);最后得出结论。
 \end{enumerate}
 \textbf{实验报告}就是预习报告、实验记录、和数据处理与分析合起来,加上本页封面。
 \item 每次完成实验后的一周内交\textbf{实验报告}。
 \item 除实验记录外,实验报告其他部分建议双面打印。
\end{enumerate}
\ 
\\
\ 

\begin{flushright}                                                           %模板作者
\tiny{
A \LaTeX \ model for General Phys. Lab., SPA, SYSU by {\em \href{https://www.weibo.com/3532532974/profile?rightmod=1&wvr=6&mod=personinfo&is_all=1}{Probfia} Gao.}\\ Adopted from the \href{http://lovephysics.sysu.edu.cn/lib/exe/fetch.php?media=courses:secondlevelzhuhai:report.docx}{original MS Word model} on \href{http://lovephysics.sysu.edu.cn}{Lovephysics}.\\ You can view it on \href{https://github.com/Probfia/SYSU_GPL_C}{Github}.}
\end{flushright}

\newpage%预习报告
\begin{center}
\LARGE{\textbf{\ExpeName}}
\end{center}
\textbf{【实验目的】}
\begin{enumerate}
 \item[1.] 熟悉微小振动测量平台的使用。
 \item[2.] 训练实验设计和操作的能力。
 \item[3.] 验证劈形薄膜等厚干涉的明、暗纹条件。
 \item[4.] 利用劈形薄膜等厚干涉测量物体的厚度。
 \item[5.] 利用等厚干涉测量微小物体的线度。
 \item[6.] 利用肌张力传感器测量单摆摆幅。
\end{enumerate}
\textbf{【仪器用具】}
%将讲义中的表格截图保存为t1在该文件夹下后删去下一行之前的%符号,合理调整scale参数。
\cpic{0.3}{t1}
%或者自己去 https://www.tablesgenerator.com/ 做一个表。
\textbf{【原理概述】}\par
该实验主要理由等厚干涉测量薄物体的厚度。\par
当光源照到一块由透明介质做的薄膜上时, 光在薄膜的上表面被分割成反射和折射两束光(分振幅),折射光在薄膜的下表面反射后,又经上表面折射,最后回到原来的媒质中, 在这里与反射光交迭,发生相干。只要光源发出的光束足够宽,相干光束的交迭区可以从薄膜表面一直延伸到无穷远。薄膜厚度相同处产生同一级的干涉条纹,厚度不同处产生不同级的干涉条纹。这种干涉称为等厚干涉,如\cref{p1}。\cpicn{0.7}{p1}{等厚干涉}
\par
劈尖夹角$\alpha \ll 1$时,在距劈尖$x$处,上表面和下表面反射光有光程差$\Delta s = 2\alpha x$,当$\Delta s = k \lambda$时,将观察到明纹;当$\Delta s = (k + \frac{1}{2}) \alpha x$时,将观察到暗纹。两相邻明纹(暗纹)间的距离为
\beq
\Delta x = \frac{\lambda}{2 \alpha} \simeq \frac{\lambda L}{2e}
\eeq
其中$e$为待测物体厚度,$L$为劈尖总长度。利用读数显微镜测出$\Delta x$的值,就可以得到待测物厚度了。
\par
反过来思考,如果用于垫高的物体厚度已知,那么,明纹间的距离也是一个已知量了,因此它可以作为一把尺子测量微小物体(亚毫米尺度)的长度。本实验中,我们利用一根$e = 0.4 \mathrm{\ mm}$的细铁丝形成劈尖,测量一个直径未知的纸上小孔的直径。达到这个目的,只需要数出小孔中条纹的个数$N$,则可以得到其直径为
\beq
d = N \Delta x = \frac{\lambda L}{2e} N
\eeq
\par
该实验的另外一部分是利用肌张力传感器测量风力和风速。将一张质量为$m$,面积为$A$的纸片挂在调零且标定好的肌张力传感器上,纸片静止时,肌张力传感器的示数是纸片的重量$mg$,若纸片受到水平方向上的恒定风力$F$,则肌张力传感器的示数将是纸片受到重力和风力的合力
\beq
T= \sqrt{(mg)^2 + F^2}
\eeq
利用一开始测到的纸片质量就可以得到水平风力$F$
\beq
F = \sqrt{T^2 - (mg)^2}
\eeq
假设风与纸片相互作用后沿平行纸面的方向扩散,那么,当摆角不大时,一个长度为$v \Delta t$的空气柱具有的动量为$ \Delta p = \rho A v \Delta t \times v = \rho v^2 A \Delta t$。根据动量定理可以算出风力
\beq F = \frac{\Delta p }{\Delta t} = \rho A v^2\eeq
从上式也可以反解,根据风力的测量值得到风速的估计。



\newpage%实验记录
\cpic{0.255}{e2}%学生信息表格
\begin{center}
\LARGE{\textbf{\ExpeName}}
\end{center}
\textbf{【实验内容、步骤、结果】}
\\
1. \textbf{利用劈尖薄膜等厚干涉测定头发丝直径}\par
将叠在一起的两块平板玻璃的一端插入一个薄片或细丝,则两块玻璃板间即形成一空气劈尖,当用单色光垂直照射时,在劈尖薄膜上下两表面反射的两束光也将发生干涉,呈现出一组与两玻璃板交接线平行且间隔相等、明暗相间的干涉条纹。
\begin{enumerate}
\item[a.] 将被测薄片或细丝夹于两玻璃片之间,用读数显微镜进行观察,描绘劈尖干涉的图像;
\item[b.] 测量劈尖的两块玻璃板交线到待测薄片间距$L$,数据记录进\cref{table0};
\begin{table}[H]
\centering
\caption{劈尖长度$L$的测量}
\label{table0}
\begin{tabular}{|c|p{12mm}|p{12mm}|p{12mm}|}
\hline
测量次数 & 1 & 2 & 3  \\ \hline
劈尖长度$L$ &  &  &   \\ \hline
\end{tabular}
\end{table}
\item[c.] 移动读数显微镜,每隔$5$个暗纹记录读数头移动的距离,数据记录进\cref{table1},进而验证等厚干涉下干涉条纹均匀分布的理论预言,并算出一个条纹间距$\Delta x$;
\item[d.] 利用公式$e = \frac{\lambda L}{2 \Delta x}$得到头发丝的厚度;
\item[e.] 用同样方法测量另外一名同学的头发丝厚度,比比看谁的头发粗,相关数据记录进\cref{table2}。
\end{enumerate}
\begin{table}[H]
\centering
\caption{读数头移动距离与暗纹级数间的关系}
\label{table1}
\begin{tabular}{|c|p{10mm}|p{10mm}|p{10mm}|p{10mm}|p{10mm}|p{10mm}|p{10mm}|p{10mm}|}
\hline
暗纹级数$k$ &  &  &  &  &  &  &  &  \\ \hline
读数头移动距离$s$ &  &  &  &  &  &  &  &  \\ \hline
\end{tabular}
\end{table}
\begin{table}[H]
\centering
\caption{另一位同学头发的读数头移动距离与暗纹级数间的关系}
\label{table2}
\begin{tabular}{|c|p{10mm}|p{10mm}|p{10mm}|p{10mm}|p{10mm}|p{10mm}|p{10mm}|p{10mm}|}
\hline
暗纹级数$k$ &  &  &  &  &  &  &  &  \\ \hline
读数头移动距离$s$ &  &  &  &  &  &  &  &  \\ \hline
\end{tabular}
\end{table}

2. \textbf{利用等厚干涉测量小孔厚度}
\par
在一张黑色纸上用绣花针戳一个较小的孔,放在劈尖下玻璃板上。利用一细铁丝垫起劈尖,数出小孔内的条纹数,重复测量,数据记录进\cref{table3}
\begin{table}[H]
\centering
\caption{小孔内的条纹数}
\label{table3}
\begin{tabular}{|c|p{12mm}|p{12mm}|p{12mm}|}
\hline
测量次数 & 1 &  2& 3  \\ \hline
孔内条纹数$N$ &  &  &   \\ \hline
\end{tabular}
\end{table}
\par
测量此时的劈尖长度,数据记录进\cref{table5}
\begin{table}[H]
\centering
\caption{劈尖长度$L$的测量}
\label{table5}
\begin{tabular}{|c|p{12mm}|p{12mm}|p{12mm}|}
\hline
测量次数 & 1 & 2 & 3  \\ \hline
劈尖长度$L$ &  &  &   \\ \hline
\end{tabular}
\end{table}
利用螺旋测微计测量细铁丝的直径,数据记录进\cref{table4}
\begin{table}[H]
\centering
\caption{细铁丝直径的测量}
\label{table4}
\begin{tabular}{|c|p{12mm}|p{12mm}|p{12mm}|}
\hline
测量次数 & 1 & 2 & 3  \\ \hline
细铁丝直径$e/\mathrm{mm}$ &  &  &   \\ \hline
\end{tabular}
\end{table}
利用上面的数据,最终可以计算出小孔的直径。
\emptyline
3. \textbf{利用肌张力传感器测量风力和风速}\par
用毫米尺测量长方形纸片的长宽如\cref{table6}
\begin{table}[H]
\centering
\caption{长方形纸片的长宽测量}
\label{table6}
\begin{tabular}{|c|p{12mm}|p{12mm}|p{12mm}|}
\hline
测量次数 & 1 & 2 & 3  \\ \hline
长方形长度$a/\mathrm{mm}$ &  &  &   \\ \hline
长方形宽度$b/\mathrm{mm}$ &  &  &   \\ \hline
\end{tabular}
\end{table}
用标准质量块标定肌张力传感器如\cref{table7}。
\begin{table}[H]
\centering
\caption{肌张力传感器的标定}
\label{table7}
\begin{tabular}{|c|p{10mm}|p{10mm}|p{10mm}|p{10mm}|p{10mm}|p{10mm}|p{10mm}|p{10mm}|}
\hline
质量块总质量$m/\mathrm{g}$ &  &  &  &  &        \\ \hline
肌张力传感器示数$V\mathrm{/mV}$ &  &  &  &  &        \\ \hline
\end{tabular}
\end{table}
\par
挂上纸片,此时肌张力传感器的示数为$V =\uline{\hspace{2cm}} \mathrm{mV}$。
\par
让一位同学对着纸片恒定地吹气,另一名同学每隔$2\mathrm{\ s}$记录下肌张力传感器的示数如\cref{table8}。
\begin{table}[H]
\centering
\caption{水平风作用下肌张力传感器的示数}
\label{table8}
\begin{tabular}{|c|p{12mm}|p{12mm}|p{12mm}|p{12mm}|p{12mm}|}
\hline
时间$t \mathrm{/s}$ & 2 & 4& 6 & 8 & 10  \\ \hline
传感器示数$V'/\mathrm{mV}$ &  &  &  &  & \\ \hline
\end{tabular}
\end{table}
\par
交换两名同学的角色重复实验,记录如\cref{table9}。
\begin{table}[H]
\centering
\caption{水平风作用下肌张力传感器的示数}
\label{table9}
\begin{tabular}{|c|p{12mm}|p{12mm}|p{12mm}|p{12mm}|p{12mm}|}
\hline
时间$t \mathrm{/s}$ & 2 & 4& 6 & 8 & 10  \\ \hline
传感器示数$V'/\mathrm{mV}$ &  &  &  &  & \\ \hline
\end{tabular}
\end{table}
最后比一比谁吹的气速度大,持续时间久。
\emptyline
\textbf{【实验过程中遇到问题记录】}

%生成最终报告时将上面内容全部删除或注释(用\iffalse \fi),将扫描得到的实验报告保存为Record.pdf在LaTeX model for GPL下,将下行命令的注释号删去。注意根据实际页数调整pages参数。
%\includepdf[pages=1-3]{Record}

\newpage%分析与讨论
\cpic{0.255}{e3}%学生信息表格
\begin{center}
\LARGE\textbf{{\ExpeName}}
\end{center}
\textbf{【分析与讨论】}\par
(Content)
\newline
\textbf{【实验思考题】}\par
(Content)

\end{document}