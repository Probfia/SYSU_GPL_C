\documentclass[CJK]{beamer}
\input{macros.tex}
\author{}
\date{}


\begin{document}

\begin{frame}
 
\begin{center}
\bch
\begin{Large}
设计性实验

\skipline
基于微小振动测量平台



\end{Large}
\skipline
莫宗霖   16308086 \\
高寒     17353019


gaoh26@mail2.sysu.edu.cn \\
mozlin@mail2.sysu.edu.cn
\ech
\end{center}
\end{frame}


\section{The Equipments}

\begin{frame}
\chtitle{仪器用具}
\bch
\begin{enumerate}
\item 读数显微镜:     测量范围$50{\rm \ mm}$,放大率$30\times$,最小读数$0.01mm$
                        测量精度$\leq0.02{\rm \ mm}$,目镜筒$360^\circ$可调,可调式半反镜
\item 测量显微镜:     目镜放大率$10X\times$,目镜测微尺$0-8{\rm \ mm}$,测微鼓轮最小分度
                        值$0.01{\rm \ mm}$,物镜放大率$2\times$,系统放大率$20\times$
\item 千分尺:        测量范围$0-25{\rm \ mm}$,最小分辨率$0.001{\rm \ mm}$,准确度$4{\rm \ \mu m}$
\item 半导体激光器:  工作电压$5{\rm \ V}$,波长$650{\rm \ nm}$
\item 劈尖:         $48{\rm \ mm}\times25{\rm \ mm}$
\end{enumerate}
\ech
\end{frame}

\section{The Principles}
\begin{frame}
\chtitle{原理概述}
\bch
该实验主要理由等厚干涉测量薄物体的厚度。\par
当光源照到一块由透明介质做的薄膜上时, 光在薄膜的上表面被分割成反射和折射两束光(分振幅),折射光在薄膜的下表面反射后,又经上表面折射,最后回到原来的媒质中, 在这里与反射光交迭,发生相干。只要光源发出的光束足够宽,相干光束的交迭区可以从薄膜表面一直延伸到无穷远。薄膜厚度相同处产生同一级的干涉条纹,厚度不同处产生不同级的干涉条纹。这种干涉称为等厚干涉。
\ech
\end{frame}

\begin{frame}
\chtitle{原理概述}
\bch
劈尖夹角$\alpha \ll 1$时,在距劈尖$x$处,上表面和下表面反射光有光程差$\Delta s = 2\alpha x$,当$\Delta s = k \lambda$时,将观察到明纹;当$\Delta s = (k + \frac{1}{2}) \alpha x$时,将观察到暗纹。两相邻明纹(暗纹)间的距离为
$$
\Delta x = \frac{\lambda}{2 \alpha} \simeq \frac{\lambda L}{2e}
$$
其中$e$为待测物体厚度,$L$为劈尖总长度。利用读数显微镜测出$\Delta x$的值,就可以得到待测物厚度了。
\ech
\end{frame}

\begin{frame}
\chtitle{原理概述}
\bch

\includegraphics[width=4in]{p1.png}

\ech
\end{frame}


\section{The Plan}
\begin{frame}
\chtitle{利用劈尖薄膜等厚干涉测定头发丝直径}
\bch
将叠在一起的两块平板玻璃的一端插入一个薄片或细丝,则两块玻璃板间即形成一空气劈尖,当用单色光垂直照射时,在劈尖薄膜上下两表面反射的两束光也将发生干涉,呈现出一组与两玻璃板交接线平行且间隔相等、明暗相间的干涉条纹。
\ech
\end{frame}

\begin{frame}
\chtitle{具体步骤}
\bch

\begin{enumerate}
\item[a.] 将被测薄片或细丝夹于两玻璃片之间,用读数显微镜进行观察,描绘劈尖干涉的图像;
\item[b.] 测量劈尖的两块玻璃板交线到待测薄片间距$L$;
\item[c.] 移动读数显微镜,每隔$5$个暗纹记录读数头移动的距离,进而验证等厚干涉下干涉条纹均匀分布的理论预言,并算出一个条纹间距$\Delta x$;
\item[d.] 利用公式$e = \frac{\lambda L}{2 \Delta x}$得到头发丝的厚度;
\item[e.] 用同样方法测量另外一名同学的头发丝厚度,比比看谁的头发粗!
\end{enumerate}
\ech
\end{frame}



\begin{frame}
\chtitle{谢谢大家!}
\bch
\begin{center}
Merci Beaucoup!
\end{center}
\ech
\end{frame}


\end{document}



